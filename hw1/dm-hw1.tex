%%%%%%%%%%%%%%%%%%%%%%%%%%%%%%%%%%%%%%%%%%%%%%%%%%%%%%%%%%%%%%
%                                                            %
%                    Read constant commands                  %
%                                                            %
%%%%%%%%%%%%%%%%%%%%%%%%%%%%%%%%%%%%%%%%%%%%%%%%%%%%%%%%%%%%%%

% Logo 

\newcommand{\dmFiveTwelve}{../resources/images/logo/data-mining}

% Problem 7

\newcommand{\examplePlot}{"../resources/images/general/problem 7/plot of precipitation, wind speed, pressure, weather symbol and UV trend over 24 hours"}
\newcommand{\exampleConsole}{"../resources/images/general/problem 7/console outputs of the example"}



\input{../resources/constants/images path/IMAGE_NAMES.tex}
\input{../resources/constants/images path/IMAGE_REFS.tex}
\input{../resources/constants/configuration/GENERAL_CONFIGURATION.tex}

%%%%%%%%%%%%%%%%%%%%%%%%%%%%%%%%%%%%%%%%%%%%%%%%%%%%%%%%%%%%%%
%                                                            %
%                       Custom Commands                      %
%                                                            %
%%%%%%%%%%%%%%%%%%%%%%%%%%%%%%%%%%%%%%%%%%%%%%%%%%%%%%%%%%%%%%

% Usage: \imageCaptionTable{image path}{image caption}{image label}{text}
% - The first argument should be the path to the image file in any format.
% - The second argument should be the caption of the image to be displayed.
% - The third argument should be the label of the image.
% - The fourth argument should be the text to be displayed on the right side of the image.
% - The resulting table will have the image displayed on the left (with the caption and label) and the text on the right.

\newcommand{\imageCaptionTable}[4]{%
	\begin{tabular}{l l}
		\includegraphics[width=0.5\textwidth, caption={#2}, label={#3}]{#1} & \centering
		\begin{varwidth}[t]{0.8\textwidth}
			\configuratedText{#4}
		\end{varwidth}
	\end{tabular}
}

% Inserts a small image in the middle of a text
% Usage: \smallimage[optional scale]{image file name}
% - The optional scale argument (default 0.5) can be used to adjust the size of the image.
% - The image file should be in the same directory as the LaTeX document.

\newcommand{\smallimage}[2][0.03]{
	\includegraphics[width=#1\linewidth]{#2}
}


% Creates a new figure with an image centered on the page, and adds a caption and label for reference.
% Usage: \sectionCenteredfigure[optional scale]{image file name}{caption text}{label}
% - The optional scale argument (default 0.9) can be used to adjust the width of the image.
% - The image file should be in the same directory as the LaTeX document.
% - The caption text should describe the contents of the image.
% - The label should be a unique identifier for the figure, used for referencing it later in the text.

\newcommand{\sectionCenteredfigure}[4][0.9]{
	\begin{figure}[H]
		\centering
		\fbox{\includegraphics[width=#1\linewidth]{#2}}
		\caption{#3}
		\label{fig:#4}
	\end{figure}
}


% Creates a new block of justified text with a specified font and font size.
% Usage: \generalText{font family}{font size}{text}
% - The font family argument specifies the font to be used (e.g., Times New Roman, Arial, etc.).
% - The font size argument specifies the size of the font (e.g., 12pt, 14pt, etc.).
% - The text argument should contain the text to be justified.
% - The resulting block of text will be fully justified (i.e., aligned with both the left and right margins).

\newcommand{\customText}[3]{%
	\par\begingroup
	\setlength{\parindent}{0pt}%
	\linespread{1.3}%
	\fontsize{#2}{#2}%
	\fontfamily{#1}\selectfont #3%
	\par\endgroup%
}

% Creates a new block of justified text with the font size and font style of configuration file.
% Usage: \generalText{text}
% - The font family argument is specfied by \userManualSimpleTextStyle 
% - The font size argument is specified by \userManualSimpleTextSize
% - The text argument should contain the text to be justified.
% - The resulting block of text will be fully justified (i.e., aligned with both the left and right margins).

\newcommand{\configuratedText}[1]{%
	\par\begingroup
	\setlength{\parindent}{0pt}%
	\linespread{1.3}%
	\selectfont #1%
	\par\endgroup%
}

% Defines a new command for referencing figures.
% Usage: \figref{label}
% - The argument should be the label of the figure to be referenced.
% - The resulting output will be in the format "(Figure <number>)", where <number> is the number of the referenced figure.
% - The label should be defined using \label{fig:<label>} command in the figure environment.
% - The \hyperref command creates a hyperlink to the referenced figure.
% - The \ref* command produces only the number of the referenced figure, without the preceding "Figure" text.

\newcommand{\figref}[1]{(\hyperref[fig:#1]{Figure \ref*{fig:#1}})}

% Creates a new table with an icon and its name.
% Usage: \customTable{icon path}{icon name}
% - The first argument should be the path to the icon file in PNG format.
% - The second argument should be the name of the icon to be displayed.
% - The resulting table will have the icon displayed on the left and its name on the right.

\newcommand{\iconNameTable}[2]{%
	\begin{tabular}{l l}
		\includegraphics[width=0.03\textwidth]{#1} & \centering 
		\begin{varwidth}[t]{0.8\textwidth}
			\configuratedText{#2}
		\end{varwidth}
	\end{tabular}
}

% Creates a new table with an icon and its name.
% Usage: \customTable{icon path}{icon name}
% - The first argument should be a text.
% - The second argument should be a text.
% - The resulting table will have the text displayed on the left and a text on the right.

\newcommand{\textTextTable}[3][2cm]{%
	\begin{tabular}{p{#1} p{\dimexpr0.90\textwidth-#1}}
		\configuratedText{#2}
		&
		\begin{varwidth}[t]{\linewidth}
			\configuratedText{#3}
		\end{varwidth}
	\end{tabular}
}

% Creates a new table with an icon, its name, and its description.
% Usage: \iconNameDescriptTable{icon path}{icon name}{icon description}
% - The first argument should be the path to the icon file in PNG format.
% - The second argument should be the name of the icon to be displayed.
% - The third argument should be a description of the icon.
% - The resulting table will have the icon displayed on the left, its name in the middle, and its description on the right.
% - The table has three columns with widths of 0.1, 0.3, and 0.5 times the text width, respectively.
% - The second and third columns are aligned to the left.

\newcommand{\iconNameDescriptTable}[3]{%
	\begin{tabular}{p{0.05\textwidth} p{0.2\textwidth} m{0.6\textwidth}}
		\includegraphics[width=0.03\textwidth]{#1} & \raggedright \configuratedText{#2} & \justify \configuratedText{#3} \
	\end{tabular}
}

% Creates a new table with an text, its name, and its description.
% Usage: \textDescriptTable{text}{text name}{text description}
% - The first argument should be the text to be displayed on the left.
% - The second argument should be the name of the text.
% - The third argument should be a description of the text.
% - The resulting table will have the text displayed on the left, its name in the middle, and its description on the right.
% - The table has three columns with widths of 0.1, 0.3, and 0.5 times the text width, respectively.
% - The second and third columns are aligned to the left.

\newcommand{\textDescriptTable}[3]{%
	\begin{tabular}{m{0.1\textwidth} m{0.1\textwidth} m{0.5\textwidth}}
		\raggedright \configuratedText{#1} & \raggedright \configuratedText{#2} & \raggedright \configuratedText{#3} \
	\end{tabular}
}

% Creates a new block of two columns with an image on the left and justified text on the right.
% Usage: \twocolumns{optional scale}{image file name}{caption text}{font family}{font size}{text}

% - 1: The optional scale argument (default 0.9) can be used to adjust the width of the image.
% - 2: The image file should be in the same directory as the LaTeX document.
% - 3: The caption text should describe the contents of the image.
% - 4: The font family argument specifies the font to be used (e.g., Times New Roman, Arial, etc.).
% - 5: The font size argument specifies the size of the font (e.g., 12pt, 14pt, etc.).
% - 6: The text argument should contain the text to be justified.

\newcommand{\twoColumns}[6]{
	\begin{minipage}[t]{0.50\textwidth}
		\customText{#4}{#5}{#6}
	\end{minipage}\hfill
	\begin{minipage}[r]{0.45\textwidth}

	\end{minipage}
}

\documentclass[12]{article}
\usepackage[utf8]{inputenc}
\usepackage{graphicx}
\usepackage{geometry}
\usepackage{tocloft}
\usepackage{amsmath}
\usepackage{booktabs}
\usepackage{fancyhdr}
\usepackage{hyperref}
\usepackage{xcolor}
\usepackage{soul}
\usepackage{times}
\usepackage{listings}
\usepackage{url}
\usepackage{wrapfig}
\usepackage{array}
\usepackage{varwidth}
\usepackage{float}
\usepackage{titlesec}
\usepackage{ragged2e}
\usepackage{todonotes}
\usepackage{tocloft}
\usepackage{changepage}

\cftsetindents{section}{1.5em}{5.0em}
\cftsetindents{subsection}{2em}{5.0em}
\cftsetindents{subsubsection}{3em}{5.0em}

% Use the new command to set the font sizes and titleformat settings
\myheadingstyles

% Customize hyperlinks in the document
\hypersetup{
	colorlinks=true, % enable colored hyperlinks
	linkcolor=blue, % set the color of internal links to black
	filecolor=magenta, % set the color of links to local files to magenta
	urlcolor=blue, % set the color of links to URLs to blue
	bookmarks=true, % enable the creation of bookmarks in the PDF file
}

\geometry{
	a4paper, % set paper size to A4
	left=2cm, % set left margin to 2cm
	right=2cm, % set right margin to 2cm
	top=2.5cm, % set top margin to 2.5cm
	bottom=2.5cm % set bottom margin to 2.5cm
}


%%%%%%%%%%%%%%%%%%%%%%%%%%%%%%%%%%%%%%%%%%%%%%%%%%%%%%%%%%%%%%
%                                                            %
%                    Document Starts Here                    %
%                                                            %
%%%%%%%%%%%%%%%%%%%%%%%%%%%%%%%%%%%%%%%%%%%%%%%%%%%%%%%%%%%%%%

\begin{document}
	
	\begin{center}
		\begin{figure}[h]
			\centering
			\includegraphics[width=0.4\linewidth]{\dmFiveTwelve}
		\end{figure}
		\vspace{1cm} 
		{\fontsize{28}{34}\selectfont \textbf{Data Mining}}
	\end{center}

	\vspace{1cm} 
	
	\begin{center}
	{\fontsize{22}{28}\selectfont \textbf{Homework \underline{1}}}
	\end{center}

	\vspace{1cm} 

	\begin{center}
	{\fontsize{22}{28}\selectfont Ashkan Ansarifard}
	\end{center}

	\vspace{0.5cm} 
	
	\begin{center}
	{\fontsize{22}{28}\selectfont 1970082}
	\end{center}

	\vspace{1cm} 
	
	\textcolor{blue!60!black}{\rule{\linewidth}{2pt}}
	
	\vspace{5cm} 
	
	\begin{center}
	\textbf{A.Y. 2022/23}
	\end{center}

	\thispagestyle{empty}
	
	\newpage
	
	\myTOC

	\newpage
	\section*{Problem 1}\label{sec:prob-1}
	\configuratedText{	
	
	\begin{enumerate}
	\item 	I choose the probability space such that each element in $\Omega$ corresponds to a unique ordering of the cards in the deck.\\
	Mathematically, $\Omega = \{\sigma_1, \sigma_2, \sigma_3, \ldots, \sigma_{52!}\}$, where each $\sigma_i$ is a distinct permutation.\\
	Since each permutation in $\Omega$ is equally likely when we shuffle a standard deck of cards, we have a uniform probability distribution.\\
	The probability measure $P$ assigns equal probabilities to all permutations in $\Omega$. Therefore, for any event $A \subseteq \Omega$, we have:\\
	\[P(A) = \frac{|A|}{52!}\]\\
	Specifically, for each individual permutation $\sigma_i$ in $\Omega$:\\
	\[P(\sigma_i) = \frac{1}{52!}\]
	\item 	Probability of the each event is:
	\begin{enumerate}
	\item \textbf{Finding the probability that the first two cards include at least one ace}
	
	We can use the complement rule:
	\[ \text{Probability (at least one ace in the first two cards)} =\]
	\[ 1 - \text{ Probability (no aces in the first two cards)} \]
	
	Now, let's calculate the probability of not getting any aces in the first two cards. There are 52 cards in the deck initially, and 48 of them are not aces. When we pick the first card, there are 48 non-ace cards out of 52 possibilities. After picking the first card, there are now 51 cards left in the deck, with 47 of them being non-ace cards. Therefore, the probability of not getting an ace on the second card, given that the first card was not an ace, is:
	\[ \frac{48}{52} \cdot \frac{47}{51} \]
	
	So, the probability that the first two cards contain at least one ace is:
	\[ 1 - \left( \frac{48}{52} \cdot \frac{47}{51} \right) \approx 0.1494 \]
	
	\item \textbf{Finding the probability that the first five cards include at least one ace}
	
	Same as item (a) we can use the complement rule:
	\[ \text{Probability (at least one ace in the first five cards)} =\]
	\[ 1 - \text{ Probability (no aces in the first five cards)} \]
	
	Same as previous item, the probability of not getting an ace on the first 5 cards, is:
	\[ \frac{48}{52} \cdot \frac{47}{51} \cdot \frac{46}{50} \cdot \frac{45}{49} \cdot \frac{44}{28}\]
	
	So, the probability that the first five cards include at least one ace:
	\[ 1 - \left(\frac{48}{52} \cdot \frac{47}{51} \cdot \frac{46}{50} \cdot \frac{45}{49} \cdot \frac{44}{28} \right) \approx 0.3147 \]
	
	\item \textbf{The first two cards are a pair of the same rank}\\
	To find the probability that the first two cards drawn from a standard deck are a pair of the same rank, you can approach it as follows:
	
	\begin{itemize}
	\item There are 52 cards in a standard deck.
	\item For the first card, there are no restrictions, so there are 52 possibilities.
	\item For the second card, you want it to be of the same rank as the first card, which means there are 3 cards of the same rank left in the deck (since there are 4 cards of each rank in a standard deck, and you've already drawn one).
	\end{itemize}
	
	So, the probability of drawing a pair of the same rank for the first two cards is:
	
	\begin{align*}
	\text{Probability} &= \frac{52/52 \cdot 3/51}{1} = \frac{1}{17}
	\end{align*}
	
	So, the probability of drawing the first two cards as a pair of the same rank is $1/17 \approx 0.0588 $.
	
	\item \textbf{The first five cards are all diamonds.}
	
	There are 13 diamonds in a standard deck.
	
	The probability of drawing a diamond on the first draw is $ \frac{13}{52} $ because there are 13 diamonds out of 52 cards.
	
	Similarly, for the second, third, fourth, and fifth draws, the probabilities are as follows:
	
	Second draw: $ \frac{12}{51} $\\
	Third draw: $ \frac{11}{50} $\\
	Fourth draw: $ \frac{10}{49} $\\
	Fifth draw: $ \frac{9}{48} $\\
	
	Because these are independent events, we can multiply these probabilities together:
	\[ \frac{13}{52} \cdot \frac{12}{51} \cdot \frac{11}{50} \cdot \frac{10}{49} \approx 0.00018184\]
	
	So, the probability of drawing the first five cards as diamonds is approximately 0.00018184, or about 0.0182\%.
	
	\item \textbf{The first five cards form a full house}\\
	There are ${4 \choose 3} = 4$ different ways to choose 3 cards of the same type and ${4 \choose 2} = 6$ different ways to choose 2 cards of the same type. To get a full house, you need to first pick the type of the 3 of a kind, which is ${13 \choose 1} = 13$ different choices, and choose the type of the pair, which is ${12 \choose 1} = 12$ different choices. The order does not count, so there are ${13 \choose 1} \cdot {4 \choose 3} \cdot {12 \choose 1} \cdot {4 \choose 2}$ ways to have a full house.
	
	\[
	{13 \choose 1} \cdot {4 \choose 3} \cdot {12 \choose 1} \cdot {4 \choose 2} = 3744
	\]
	
	Note that there are ${52 \choose 5} = 2,598,960$ different combinations for the first 5 cards, so the probability of being dealt a full house is:
	
	\[
	\frac{3744}{2,598,960} \approx 0.14\%
	\]
	
	\end{enumerate}
	\end{enumerate}
	}

	\newpage
	\section*{Problem 2}\label{sec:prob-2}
	\begin{enumerate}
	\item For calculating the probability that the baby born at midnight was a boy I design the following probability space:
	 let's define $ E_{1} $ and $ E_{2} $ as following:
	 
	 $ b $: Number of boys before midnight (In our case 4)
	 
	 $ g $: Number of girls before midnight
	 
	 $ E_{1} $: boy is born at midnight
	 
	 $ E_{2} $: girl is born at midnight\\
	
	Let say that initially there were $ n $ girls.
	
	$ B $: boy picked up by nurse
	
	$ G $: girl picked up by nurse\\
	
	\item We have to calculate $ P(E_{1}|B) $:
	
	\[
	P(E_{1}|B) = \frac{P(E_{1} \cap B)}{P(B)}
	\]
	
	\textbf{Assumption}: I assumed, when a child is born, the probability of that child being either a boy or a girl is 50\%
	
	The chance of picking a boy, which we'll call P(B), is calculated by adding together the probability of picking a boy such that the boy was born at midnight and the probability of picking a boy such that the girl was born at midnight.
	
	\[
	P(B) = \frac{1}{2} \cdot P(B|E_{1}) + \frac{1}{2} \cdot P(B|E_{2}) 
	\]
	\[
	P(B) = \frac{1}{2} \cdot \frac{b + 1}{b + 1 + g} + \frac{1}{2} \cdot \frac{b}{b + 1 + g} = \frac{1}{2} \cdot \frac{2b + 1}{b + 1 + g}
	\]\\
	
	As $ P(E_{1} \cap B) $ = $ P(B \cap E_{1}) $, and  $ P(B \cap E_{1}) $ = $ P(B | E_{1}) \cdot  P(E_{1}) \dots$
	
	\[
	P(E_{1}|B) = \frac{P(B | E_{1}) \cdot  P(E_{1})}{P(B)}
	\]
	
	So,
	
	\[
	P(E_{1}|B) = \frac{\frac{1}{2} \cdot \frac{b + 1}{b + 1 + g}}{\frac{1}{2} \cdot \frac{2b + 1}{b + 1 + g}} = \frac{b + 1}{2b + 1} = \frac{5}{9} \approx 0.56
	\]
	
	\end{enumerate}

	\newpage
	\section*{Problem 3}\label{sec:prob-3}
	Let's define the following probability space for this problem:
	
	$ n $: Number of times your roll three dices
	
	$ E_{11} $ : Sum is 11
	
	$ E_{16} $ : Sum is 16

	$ E_{11}^{C} $ : Sum is not 11
	
	$ E_{16}^{C} $ : Sum is not 16\\
		
	Let's simulate some numbers of rolling times and calculate the possible outcomes to gain an insight:
	
	\[
	n = 1 \rightarrow
	\begin{cases}
	E_{11}^{1}  & P(E_{11}) = \frac{18}{216}\\
	E_{16}^{1}  & P(E_{16}) = \frac{6}{216 }\\
	(E_{11}^{1} \textit{ and } E_{16}^{1})^{C}  & P(E_{11}^{C} \cup E_{16}^{C}) = \frac{195}{216 }
	\end{cases}
	\]

	\[
	n = 2 \xrightarrow[\text{in n = 1}]{\text{Sum was not 11 not 16}}
	\begin{cases}
	E_{11}^{2}  & P(E_{11}) = \frac{195}{216 } \cdot \frac{18}{216}  \\
	E_{16}^{2}  & P(E_{16}) = \frac{195}{216 } \cdot \frac{6}{216 } \\
	(E_{11}^{2} \textit{ and } E_{16}^{2})^{C} & P(E_{11}^{C} \cup E_{16}^{C}) = \frac{195}{216 } \cdot \frac{195}{216 }
	\end{cases}
	\]

	\[
	n = 3 \xrightarrow[\text{not n = 1 nor in n = 2}]{\text{Sum was not 11 not 16}}
	\begin{cases}
	E_{11}^{3}  & P(E_{11}) = \frac{195}{216 } \cdot \frac{195}{216 } \cdot \frac{18}{216}  \\
	E_{16}^{3}  & P(E_{16}) = \frac{195}{216 } \cdot \frac{195}{216 } \cdot \frac{6}{216 } \\
	(E_{11}^{3} \textit{ and } E_{16}^{3})^{C} & P(E_{11}^{C} \cup E_{16}^{C}) = \frac{195}{216 } \cdot \frac{195}{216 } \cdot \frac{195}{216 }
	\end{cases}
	\]\\
	
	So, we found a trend!
	
	The probability that you stop because you see a sum of 16 in the $ n^{th} $ time is:
	\[
	P(E_{16})^{n} = P(E_{11}^{C} \cup E_{16}^{C})^{n-1} \cdot P(E_{16})
	\]
	
	Which will be:
	\[
	P(E_{16})^{n} = (\frac{195}{216})^{n-1} \cdot \frac{6}{216}
	\]

	\newpage
	\section*{Problem 4}\label{sec:prob-4}
	To find the expectation of X, we need to calculate the average number of times the word ``mining'' appears in a randomly typed sequence of $100,000,000,000$ letters.\\
	
	The word ``mining'' consists of $6$ letters, and each letter has an equal probability of being typed by the monkey. So, the probability of the monkey typing the word ``mining'' in a specific order is $\left(\frac{1}{26}\right)^6$.\\
	
	Now, let's calculate the number of possible positions where the word ``mining'' can start in a sequence of $100,000,000,000$ letters. Since there are $100,000,000,000 - 6 + 1 = 99,999,999,995$ possible starting positions, the probability of the monkey typing the word ``mining'' in any of these positions is $99,999,999,995 \left(\frac{1}{26}\right)^6$.\\
	
	To find the expectation of X, we multiply the probability of the monkey typing the word ``mining'' in any given position by the total number of possible starting positions:
	
	\[
	\text{Expectation of } X = 99,999,999,995 \left(\frac{1}{26}\right)^6
	\]
	
	Let's calculate this value:
	
	\[
	\text{Expectation of } X = 99,999,999,995 \left(\frac{1}{26}\right)^6
	\]
	\[
	\text{Expectation of } X \approx 99,999,999,995 \times 3.2371e^{-9}
	\]
	\[
	\text{Expectation of } X \approx 323.71
	\]
	
	Therefore, the expectation of $X$, the number of times the word ``mining'' appears, is approximately $323.71$.
	
	\newpage
	\section*{Problem 5}\label{sec:prob-5}
	To find the probability of seeing a bicycle in a given time interval, we can assume that the probability of seeing a bicycle remains constant over time. 
	
	We already know the probability of seeing a bicycle in 45 minutes as P(45), which is 97\%. We want to find the probability of seeing a bicycle in a 15-minute interval, written as P(15).
	
	Since the probability remains constant, we can assume that the ratio of probabilities is the same as the ratio of time intervals:
	
	\[
	\frac{{P(15)}}{{P(45)}} = \frac{{15}}{{45}}\]
	
	To find P(15), we can rearrange the equation:
	
	\[P(15) = \frac{{15}}{{45}} \times P(45)\]
	
	Substituting the given value of P(45) as 97\% or 0.97:
	
	\[P(15) = \frac{{15}}{{45}} \times 0.97\]
	
	Calculating this expression:
	
	\[P(15) = 0.32\]
	
	Therefore, the probability of seeing a bicycle in a 15-minute interval is 32\%.

	\newpage
	\section*{Problem 6}\label{sec:prob-6}


	\newpage
	\section*{Problem 7}\label{sec:prob-7}
	For this problem a console program based on two libraries:
	\begin{enumerate}
	\item 	\textit{requests} Python library $ \rightarrow $ DirectAPIClient.py
	\item 	\textit{meteomatics.api} library $ \rightarrow $ HighLevelAPIClient.py
	\end{enumerate}
	
	\subsection*{Retrieving Data Using DirectAPIClient Class}
	The \texttt{DirectAPIClient} class retrieves weather data from the Meteomatics API through the following steps:
	
	\begin{enumerate}
	\item It formats the time range for the query using the provided start date, end date, and interval.
	\item The API URL is constructed by combining the base URL, time range, selected parameters, coordinates (latitude and longitude), and specifying the output format as JSON.
	\item An HTTP GET request is sent to the constructed API URL, including your Meteomatics API username and password for authentication.
	\item The response from the API is captured and checked for its status code.
	\item In the case of a successful response (status code 200), the function parses the JSON response and extracts the weather data.
	\item The weather data is processed to obtain time series for each selected parameter.
	\item A Plotly figure is created, with traces (lines) added for each parameter to visualize trends over time.
	\item The function includes error handling for errors based on the given HTTP status codes and provides informative error messages.
	\item Also exception handling is implemented to catch any unexpected exceptions during the process.
	\item The Plotly figure is displayed, allowing users to analyze and visualize the weather data.
	\end{enumerate}
	
	\newpage
	\subsection*{How to Use The Program}
	
	\configuratedText{Run Python Main.py
	
	Follow the prompts to select parameters, input dates and time intervals, and choose either the high-level or direct API client.
	
	The program will then query the Meteomatics API, retrieve weather data, and display it as a Plotly graph.
	
	You can visualize and analyze the trends of the selected weather parameters over a any time period that you want.}

	\subsection*{Example}
	In this example, Precipitation, Wind speed, Pressure, Weather Symbol and UV trend over 24 hours is plotted.

	\sectionCenteredfigure{\exampleConsole}{Console Output of the Example}{console-outputs-of-the-example}
		
	\sectionCenteredfigure{\examplePlot}{Plot of Precipitation, Wind speed, Pressure, Weather Symbol and UV trend over 24 hours}{example-plot}
	
\end{document}